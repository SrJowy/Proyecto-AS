\documentclass[12pt]{article}

\usepackage{amsmath}

\usepackage{graphicx}

\usepackage{hyperref}

\usepackage[utf8]{inputenc}

\title{Despliegue de un sistema con CrateDB}

\author{Joel Bra Ortiz}

\date{2020–12–10}

\begin{document}

\maketitle

\section{Introducción}

En este proyecto se nos pide aplicar los conocimientos recibidos en la
asignatura de Administración de Sistemas para gestionar y orquestar un
conjunto de contenedores.

En primer lugar, se nos provisto de una imagen docker de Docker Hub que
tendremos que utilizar como base, en este caso, CrateDB.

Por otro lado, deberemos incluir otras funcionalidades a la aplicación en
forma de contenedor y generar un despliegue conjunto con docker-compose.

\section{Los contenedores}

\subsection[1]{CrateDB}
CrateDB es el contenedor que me ha sido asignado. Es una base de datos SQL
construida sobre una arquitectura en la nube. Además, añade la escalabilidad
y flexibilidad de una base de datos NoSQL.



\begin{itemize}

\item One

\item Two

\item Three

\end{itemize}

\end{document}