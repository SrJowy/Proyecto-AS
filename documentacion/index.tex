\documentclass[12pt]{article}

\usepackage{amsmath}

\usepackage{graphicx}

\usepackage{hyperref}

\usepackage[utf8]{inputenc}
\usepackage{parskip}
\usepackage{listings}

\title{Despliegue de un sistema con CrateDB}

\author{Joel Bra Ortiz}

\date{2020–12–10}

\begin{document}

\maketitle

\section{Introducción}

En este proyecto se nos pide aplicar los conocimientos recibidos en la
asignatura de Administración de Sistemas para gestionar y orquestar un
conjunto de contenedores.

En primer lugar, se nos ha provisto de una imagen docker de Docker Hub que
tendremos que utilizar como base, en este caso, CrateDB\@.

Por otro lado, deberemos incluir otras funcionalidades a la aplicación en
forma de contenedor y generar un despliegue conjunto con docker-compose.

\section{Los contenedores}

\subsection[1]{CrateDB}
CrateDB es el contenedor que me ha sido asignado. Es una base de datos SQL
construida sobre una arquitectura en la nube. Además, añade la escalabilidad
y flexibilidad de una base de datos NoSQL\@.

Está escrita en Java y basada en la arquitectura shared-nothing, la cual
consiste en que cada nodo es autosuficiente e independiente. De esta forma,
ningún nodo comparte memoria o espacio de disco duro.

CrateDB cuenta con unos nodos que distribuyen y coordinan de manera
automática la ejecución de las tareas de escritura y peticiones a lo largo
del cluster.

Al tratarse de una base de datos SQL tiene a nuestra disposición JOINS, índices,
vistas\ldots Para su distribución, CrateDB cuenta con una cahé que le dicen al
motor de peticiones si existe alguna fila que cumpla con los criterios de la query.

Por otro lado, CrateDB soporta esquemas de tipo estricto, dinámico o ignorado. Al
igual que una base de datos noSQL, permite actualizar o no el esquema de una tabla
al realizar un INSERT.\@

CrateDB implementa un sistema que se contrapone al clásico sistema
ACID (atomicity, consistency, isolation, durability), siendo clasificado como 
BASE (basically-available, soft-state, eventual consistency). Cuenta con un sistema
de versiones, un sistema de concurrencia optimista (asume que varias transacciones
se pueden realizar sin interferir entre ellas) y un ajuste de actualización a nivel
de tabla que le permite ser consistente cada \emph{x} milisegundos.

En cuanto a la atomicidad y la durabilidad, las operaciones sobre filas son atómicas.
Además, éstas tendrán persistencia en disco sin realizar un commit. En caso de que se
produzca un apagado inesperado de un nodo, las operaciones del diario se vuelven a
ejecutar para asegurarse de que todas las operaciones se han realizado.

\subsection[2]{Aplicación cliente}
La aplicación conocida como cliente hará las veces de aplicación servidor que nos
conectará con la base de datos CrateDB para almacenar la información pertinente y que
esta se visualice en el frotnend.

Para la programación de la misma se ha utilizado el lenguaje Python que nos permite, por
un lado, recibir peticiones mediante la librería \emph{Flask} y, por otro, realizar operaciones
SQL en nuestra base de datos mediante el paquete \emph{crate}. La aplicación quedará estará
almacenada en el archivo \emph{app.py} que se muestra a continuación:

\begin{lstlisting}[caption={Código de app.py},captionpos=b,language=python]
from crate import client
from flask import Flask, request, jsonify
from flask_cors import CORS
import time

app = Flask(__name__)
CORS(app)

@app.route('/retrievedata',methods=['GET'])
def retrieve_data():
    connection = 
        client.connect("http://10.5.0.4:4200/", 
        username="crate")

    if connection:
        cursor = connection.cursor()
        cursor.execute("""select table_name 
            from information_schema.tables WHERE 
            table_name = 'tasks' limit 100""")
            exists = cursor.fetchone()

        if (not exists):
            cursor.execute("""CREATE TABLE tasks 
                (name text, task text, date text)""")
            cursor.execute("""COPY tasks FROM 
                'file:///db-data/tasks_1_.json'""")
            cursor.execute("""COPY tasks FROM 
                'file:///db-data/tasks_2_.json'""")
            time.sleep(1)

        cursor.execute("SELECT * FROM tasks")
        tasks = cursor.fetchall()
        connection.close()

    retorno = {"lTareas": []}
    for t in tasks:
        retorno["lTareas"]
            .append({"name": t[0], 
            "task": t[1], "date": t[2]})
    
    return jsonify(retorno)

@app.route('/newentry', methods=['POST'])
def new_entry():
    name = request.form['name']
    task = request.form['task']
    date = request.form['date']

    connection = 
        client.connect("http://10.5.0.4:4200/",
        username="crate")
    cursor = connection.cursor()
    cursor.execute("""INSERT INTO "doc"."tasks" 
        (name, task, date)  VALUES (?, ?, ?)""",
        (name, task, date))
    connection.close()

    return "Succesfull"

app.run(host='0.0.0.0', port=4201)

\end{lstlisting}

La aplicación cuenta con dos peticiones diferentes, una con método GET y otra con POST con
diferentes funcionalidades:
\begin{itemize}
    \item /retrievedata: se ejecuta cuando carga el frontend. Crea la tabla en la base de datos
    en caso de que no exista y carga los datos almacenados en el volumen. A continuación, se realiza
    una operación de SELECT mediante la cual, se recuperan los datos que después se mostrarán en el frontend.
    \item /newentry: recupera los datos del frontend y, mediante una operación de INSERT, los añade a la
    base de datos.
\end{itemize}

Para encapsular la aplicación cliente se ha creado el siguiente Dockerfile, cuyo repositorio en
Docker Hub es \url{https://hub.docker.com/r/srjowy/cliente-crate}:

\begin{lstlisting}[caption={Archivo Dockerfile}, captionpos=b]
    FROM alpine

    RUN apk -qq update
    RUN apk -qq add python3
    RUN apk -qq add py3-pip
    RUN pip3 -qq install crate
    RUN pip3 -qq install flask
    RUN pip3 -qq install flask_cors
    RUN pip3 -qq install docker
    RUN mkdir /cliente
    
    COPY ./app.py /cliente/app.py
    
    WORKDIR /cliente
    
    CMD python3 app.py
\end{lstlisting}


En primer lugar, hemos utilizado alpine como imagen base debido a su ligereza. A continuación,
instalamos las dependencias necesarias tanto con apk como con pip para las librerías que utilizamos
en la aplicación cliente. Después creamos el directorio \emph{/cliente} donde almacenamos el archivo python
que después ejecutamos para echar a correr la aplicación y que se quede escuchando para tratar las peticiones
que le lleguen.

\subsection[3]{Frontend}

La tercera aplicación implementada es un servidor web. Para implementarlo hemos utilizado la imagen docker
\emph{httpd}. Esta aplicación cuenta con un sencillo archivo html que cuenta con tres etiquetas de tipo
input que permiten escribir y cuya información se procesará en un archivo javascript que realizará las
pertinentes peticiones al backend (aplicación cliente) para almacenar o mostrar la información.

\section{Archivo docker-compose}
Para lograr que la aplicación tenga sentido en su conjunto, todos los contenedores deben comunicarse entre
si. Para ello, se ha creado un archivo docker-compose que gestionará la red y los volúmenes necesarios para el
correcto funcionamiento de la base de datos y el frontend.

\begin{lstlisting}[caption={Archivo docker-compose.yml}, captionpos=b]
services:
  servidor-bd:
    image: crate
    ports:
      - 4200:4200
    volumes:
      - ./db:/db-data
    command: ["crate",
              "-Cdiscovery.type=single-node",
              "-Cnetwork.host=_site_"]
    hostname: crate-db
    networks:
      my-network:
        ipv4_address: 10.5.0.4
  cliente-crate:
    build: .
    ports:
      - 4201:4201
    networks:
      my-network:
        ipv4_address: 10.5.0.5
    depends_on:
      - servidor-bd
  web:
    image: httpd
    volumes:
      - ./public:/usr/local/apache2/htdocs
    ports:
      - 8080:80
    networks:
      my-network:
        ipv4_address: 10.5.0.6
    depends_on:
      - cliente-crate
networks:
  my-network:
    driver: bridge
    ipam:
      config:
        - subnet: 10.5.0.0/16
          gateway: 10.5.0.1
volumes:
  db-data:
\end{lstlisting}

El archivo docker-compose cuenta con tres servicios que representan las tres partes de
la aplicación. Cada uno tiene una dirección ipv4 que le permitirá identificarse dentro
de la red \emph{my-network}.

Por otro lado, cada servicio cuenta con sus dependencias para que todo se ponga en marcha
a la vez y sin problemas de conexiones.

En cuanto a la persistencia, se ha creado un volumen al que hemos llamado \emph{db-data} en
el que almacenaremos las entradas de la base de datos. Además, copiaremos los datos de la carpeta
public a la carpeta htdocs del contenedor web para que tenga disponible los archivos html
y javascript del frontend.

\section{Kubernetes}
Para realizar la implementación de Kubernetes debemos tener en cuenta los objetos necesarios
para que la aplicación se ejecute de la misma forma que lo hace con el archivo docker-compose
pero con los beneficios de utilizar Kubernetes. Para cada aplicación hemos creado un objeto
deployment, con su determinado cluster-ip para que cada contenedor sea visible dentro del cluster
y para los contenedores que necesitan persistencia, persistentVolumeClaims.

Finalmente, para la conexión con el exterior, hemos creado un objeto de tipo ingress.

\end{document}